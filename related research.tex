\documentclass[12pt]{ctexart}
\usepackage[T1]{fontenc}
%\usepackage[spanish]{babel} 
%\spanishdecimal{.}
\usepackage[utf8]{inputenc}
\usepackage{helvet}
\renewcommand{\familydefault}{\sfdefault}
\usepackage{amsmath,amsthm,amsfonts,amssymb,amscd}
\usepackage{multirow,booktabs}
\usepackage[table]{xcolor}
\usepackage{fullpage}
\usepackage{lastpage}
\usepackage{enumitem}
\usepackage{fancyhdr}
\usepackage{mathrsfs}
\usepackage{wrapfig}
\usepackage{setspace}
\usepackage{calc}
\usepackage{multicol}
\usepackage{cancel}
\usepackage[retainorgcmds]{IEEEtrantools}
\usepackage[left=2cm,right=2cm,top=2cm,bottom=3cm]{geometry}
\usepackage{floatrow}
\DeclareMathOperator\argmin{argmin}
\DeclareMathOperator\eg{e.g.}
\DeclareMathOperator\ie{i.e.}
\DeclareMathOperator\R{\mathbb{R}}
\DeclareMathOperator\Fourier{Fourier}
\newlength{\tabcont}
\setlength{\parindent}{0.0in}
\setlength{\parskip}{0.05in}
\theoremstyle{definition}
\newtheorem{defn}{Definición}
\newtheorem{reg}{Regla}
\newtheorem{ejer}{EJERCICIO}
\pagestyle{fancyplain}
\headheight 16pt
\newcommand\course{link prediction}	
\lhead{\course}
\chead{\textbf{related researches}}
\rhead{Xincan FENG}
%---------------cheat sheet---------------
%\begin{itemize}
%\item 
%\begin{vmatrix}
%1& 1 & 0&0\\ 
%0& 0 & 1&1\\ 
%1& 0 & 1&0\\
%0& 1 & 1&1
%\end{vmatrix}$\\
%\item $\vec{B}= \left\lbrace\ \begin{bmatrix}
%1& 1\\ 
%0& 0
%\end{bmatrix},\begin{bmatrix}
%0& 1\\ 
%1& 1
%\end{bmatrix}\right\rbrace\ $
%\end{itemize}
%---------------cheat sheet---------------
\begin{document}
\section{引言}
\subsection{现有研究}
Resource Description Framework (RDF):
从思维到自然语言到图谱,是人们对智慧的整理过程。\\
互联网内容的大规模、异质多元、组织结构松散的特点,
给人们有效获取信息和知识提出了挑战。
现在的大型知识图谱以其强大的语义处理能力和开放组织能力,
为互联网时代的知识组织和智能应用奠定了基础。
但知识图谱能否直接帮助用户管理自己的知识呢?

\subsection{本研究}
\subsubsection{思想和想做的事}
跨领域(包括跨语言)的重要性:\\
从现有研究来看,我关注到了知识图谱应用的一个问题:共用还是私用?\\
共用数据需要达到的效果侧重于:link prediction(增强搜索功能),
kg主要在做整理知识的工作。比起用知识专家,让AI用写好的知识进行传播效率更高。\\
私用数据则侧重于:知识抽取、表示、融合、推理,
kg主要在做挖掘资源的工作。\\
但比起知识专家,AI专家的作用有多少呢?
即便AI一直在给予知识专家辅助,但知识专家永远可以指责AI专家给的知识没有用。
除非AI脱离了人类智慧,否则这一指责永远不会失败。\\
人类对事物及其关系的认知是个黑匣子,AI是对这个黑匣子的量化学习。
向一个专家学习,AI或许只能成为效能为10\%的智能专家,
但向100个专家学习,最大可达到1000\%的效能——“跨领域学习”是让AI应用强势的关键。\\

强调路径而非entity的重要性:\\
另一方面,所谓思维,就是找到解决问题最近的路。
AI胜在可以同时考虑各种已知因素、从已知路径中找到最近的,
人的思维在于懂得创造,如果可能的话,他们能创造出新的路——捷径。
而在实际问题中,由于状况千变万化,大部分时候都需要这样一些“创新”。
AI能创造“捷径”吗?\\
现在,AI与人类的关系看起来就好像父母与孩子,
当孩子懵懂无知时,AI提供的知识就可以让人受益匪浅,是最不偏心、偏颇的知识提供者;
当孩子成为了专家可以独当一面时,AI所提供的知识开始显得力不从心。\\
但AI所提供的重点不是知识本身,而是路径,AI是最好的引领者。
将来AI和人类的关系,可以是DNA和人类的关系。
它能将信息分割成合适的单元,对这种单元的新组合(未知信息),也有早就写好的计算法则应对,
不断接受信息的同时,因为反应迅速而一直站在指导人类的地位。
这对人类专家也是有好处的,顾虑周全的AI的可以让人类专家更“专”更“快”。\\
我想提出的概念,不是知识图谱,而是知识路径。
因为人的任何一步,都受到了细思极恐般的无限的限制:
生命、记忆、思维能力、技能、环境……
总的来说,就是自己的位置和各种资源总量的限制。
这种给予本体的推理对于资源预测和挖掘即位重要。\\
“先适应”和“最适化”的区别在于后者是计算现有条件下的最优选择,即现在的结果是最优;
而“先适应”是指使选择做出后、下一步的结果最优。
(但二者都属于自适应算法self-adaptive的研究)\\

强调end-to-end的重要性:\\
让数据能自我判断的空间更大,简化和优化人工操作。

强调auto-structured的难点:\\
人类脑中的世界是continous的,但语言是discrete。
sturcture data本身是把语言更加structure化、
强structure化,对专家而言仍然会被理解、且能节省时间,
但对普通人而言,会使内容本身更难以被理解。\\
理解,一部分是对文字意思但理解,更多的则是对其组成和相互作用的理解。
前者称为语义,后者称为知识。
机器可以很好的理解明确定义的语义,但很难理解没有明确定义的知识。\\
一开始是不存在概念的,当区别出现后,就自然会聚类出概念以分类。

\subsection{目标}
“学以致用”,对于机器更是如此,因此,我认为以下AI应用研究很具有未来性:
end-to-end跨领域(包括跨语言)的知识图谱构建、
和各资源限制下的先适应知识路径提供(基于本体推理ontology reasoning)。\\
研究的指导思想是:让自己时刻都是被需要的。\\
我的应用目标是,能自动挖掘资源的kg:vita.\\
1. 它是全方位的活着的人类记忆(思维,用与kg配套的立体空间space表示)。
有这个目标的原因是,我相信总有一天,人类被虚假和针对性营销包围,
会渴望找到和立体化自己的记忆的。那时的AI就不再只是工具和应用,而是结果和需求本身了。\\
2. 它也是抽象但足够被人类理解的语言(kg)。
人对entity的理解就是对物的理解,但人希望了解的其实是事,所以所谓“事”该如何表示?
就好像DNA一样,有了单词,如何连线呢?连成线的entity,就成为了“事”。
物是客观证据,事是人为想象,人和机器的理解(认知和表现力)会因所依据的kg不同而改变。
即便不能比现有的自然语言更丰富多彩,但人类可以把语言这门艺术简化、或者说抽象化,
形成一种更容易达成共识的机器语言——kg语言。
这种语言会是编程语言和自然语言的友善过渡,也是世界通用的充满智慧的语言。
而掌握这种语言的nlp专家,将担任未来知识的导航者、甚至是先知,
因为未来的知识必将更加庞大而精细,超出任何人的时间限制,
而nlp专家必须保证能领导人们拥有担任某项工作足够的知识。

\subsubsection{方法}
试图做的研究:\\
多源抽取融合:用词向量相加和CNN两种方法分别考虑文本的拓展表示和词序信息,
比如DKRL可以坐kbc知识融合,再比如利用跨语言优势高效学习。\\
复杂关系表示:但不同的知识要有意义,所需的kg结构并不相同,比如树状或网状,
所以融合手段是有限的,且最好要有机抽取融合。
比如distant supervision, open information extraction, CNN\\
path-infer:路径可以用来表达语义吗?
相加、位乘、RNN哪种算法可以提高path的可靠性?目前是相加最好。
比如path-constraint random walk, path ranking algorithm,
path-based TransE (PTransE)利用path预测关系\\

试图做的应用:\\
制作可视化工具或算法,让embedding更直观\\
用kg和embedding直接理解文章\\
用kg做私人活动监视器,让人们了解自己的信息摄入状况\\
用kg尝试做优于人类精算师的保险计算\\

\section{embedding model with more rationality and interpretability}
\subsection{learning of existing methods}
独热表示one-hot之后,现在主要研究的是分布式表示distributed representation,
难点在于增强实体和关系之间的联系。\\
为便于理解和比较不同模型的想法,以下尽量采用了统一的表示方法。
其中,map所表示的目前都是使用linear map进行计算的,
而non-linear map的部分都用特定函数直接表示出来了。
即我所写的map部分其实也可用matrix表示,
之所以不写成matrix,是因为考虑到将来的研究中,
现在的linear map也可能拓展到non-linear map,
所以,我只特别表示出了non-linear map的部分,而其它部分则仍用map表示:
\[\begin{array}{l}
    entities\in \R^d\\
    knowledge=(entities,links,triples)\\
    triple=(head,link,tail)\\
    triple^-=(head,link^-,tail) \bigcap (head^-,link, tail) \bigcap (head, link, tail^-)
\end{array}\]

\subsubsection{距离模型}
structured embedding (SE):重要缺陷:head,tail的map用两个不同的矩阵,协同性较差。
\[\begin{array}{l}
    map_1, map_2\in \R^{d\times d}\\
    \begin{array}{lll}
        link&=&{\argmin}\,distance\\
        &=&{\argmin}\,|map_1\times head - map_2\times tail|_{L_1}\\
    \end{array}
\end{array}\]

\subsubsection{单层神经网络模型}
single layer model (SLM):只增强了微弱的联系,且$\tanh$的计算复杂度高。
\[\begin{array}{l}
    link^T\in \R^k\\
    map_1, map_2\in \R^{d\times k}\\
    link={\argmin}\,link^T \times \tanh(map_1\times head+map_2 \times tail)
\end{array}\]

\subsubsection{语义匹配能量模型}
semantic matching energy (SME):也是为了增强联系,计算更复杂了。
\[\begin{array}{l}
    \otimes: A_{ij}\otimes B_{ij}=\{a_{ij}\times b_{ij}\},\mathrm{Hadamard\,product}\\
    \begin{array}{lll}
        link&=&{\argmin}\,(map_1\times head+map_2 \times link+bias_1)^T\\
        &&(map_3\times tail+map_4 \times link+bias_2)\\
        link&=&{\argmin}\,(map_1\times head\otimes map_2 \times link+bias_1)^T\\
        &&(map_3\times tail\otimes map_4 \times link+bias_2)\\
    \end{array}
\end{array}\]

\subsubsection{双线性隐变量模型}
Latent factor model (LFM): 协同性较好,计算复杂度低。
\[\begin{array}{l}
    map\in \R^{d\times d}\\
    link={\argmin}\,head^T\times map\times tail
\end{array}\]

Dismult:计算大大减小,效果反而显著提升。
\[\begin{array}{l}
    map\in D^{d\times d}, \mathrm{diagonal\,matrix}
\end{array}\]

\subsubsection{张量神经网络模型}
Neural tensor network (NTN):这里的实体向量是该实体中所有单词向量的平均值。
这样可以充分重复利用单词向量,降低稀疏性问题,但计算复杂度非常高,难以扩大规模。
\[\begin{array}{l}
    map_t \in \R^{d\times d\times k}, \mathrm{third-order \,tensor}\\
    map_1, map_2 \in P^{d\times k}, \mathrm{projection \,matrix}\\
    link=map^T \times \tanh(head \times map_t \times tail+map_1\times head+map_2\times tail+bias)
\end{array}\]

\subsubsection{矩阵分解模型}
RESACL:不仅优化存在link的位置,也优化不存在link的位置。
\[\begin{array}{l}
    \exists\, link: head\times map \times tail=1\\
    \nexists\, link: head\times map \times tail=0\\
    \ie\\
    link={\argmin}\, head\times map \times tail-1\\
    link^-={\argmin}\, head\times map \times tail
\end{array}\]

\subsubsection{翻译模型}
TransE:根据平移不变现象,捕捉单词之间相同的隐含语意关系,
相当于用相对的link去解释/翻译单词。
随机替换$triple$的人一元素以创建$triple^-$,继续考虑了错误link的情况。
参数较少,计算复杂度低,在大规模稀疏kg上效果尤其惊人。
因此成为以后大部分kge的基础模型。
\[\begin{array}{l}
    head+link=tail\\
    \eg\\
    C(king)-C(queen)\approx C(man)-C(woman)\\
    \ie\\
    link={\argmin}\, |head+link-tail|_{L_1\mathrm{or}L_2}\\
    \mathrm{for\,further\,optimization,}\\
    triple^-=(head,link^-,tail) \bigcap (head^-,link, tail) \bigcap (head, link, tail^-)\\
    Link=\sum\limits_{triple} \sum\limits_{triple^-} \max (0,link-link^-+triple-triple^-)
\end{array}\]

Holographic embeddings (Hole):让head和tail循环相关circular correlation 
(an operator that combines the expressive power of the tensor product with the efficiency and simplicity of TranE.)
此操作的优点:
1)不可交换性,许多kg中的link都是不可交换的;
2)相关性,循环相关的得到的向量每一维都限制了head和tail的某种相似性,例如第一维相当于head和tail的内积,这利于更清楚地分清比较相似的关系;
3)计算效率高,可用快速傅里叶变换加速计算。
\[\begin{array}{l}
    \star: \R^d\times \R^d\rightarrow \R^d, \mathrm{circular\, correlation.}\\
    {[head\star tail]}_k=\sum\limits_{i=0}^{d-1} head_i \times tail_{(i+k)\mod d}\\
    \mathrm{for\,further\,optimization,}\\
    head\star tail= \Fourier^{-1}\left(\overline{\Fourier}(head)\odot \Fourier(tail)\right)\\
    link=\argmin \sigma (link^T(head\star tail))
\end{array}\]

TransH:处理1-N,N-1,N-N的复杂关系。如果map后存在无限个link超平面,则选取与map近似正交的link超平面。
\[\begin{array}{l}
    map: \mathrm{normal\, vector\, of\, head\, and\, tail}\\
    link=\argmin |(head-map^T \times head \times map) + link - (tail-map^T \times tail \times map)|_{L_1 \mathrm{or} L_2}
\end{array}\]

TransR:使实体和关系可以map到不同的space,再在不同space中建立link。
\[\begin{array}{l}
    map\in \R^{d\times k}\\
    link=\argmin |head\times map + link - tail\times map|_{L_1\mathrm{or} L_2}
\end{array}\]

CTransR:对head和tail的差进行聚类,从而将link细分成子关系$link_{child}$,每个子关系分别学习向量表示。
\[\begin{array}{l}
    link=\argmin |head\times map + link_{child} - tail\times map|_{L_1\mathrm{or} L_2}
\end{array}\]

TransD:认为让map仅和link有关系是不合理的,还是应该和entity有关;且TransR引入不同space导致参数急剧增加、计算复杂度大大提高。因此TransD设计了两个map,让map与entity和link都相关,同时减小计算复杂度。
\[\begin{array}{l}
    head_{project}, tail_{project}\in \R^d\\
    link_{project}\in \R^k\\
    map_{head}=link_{project} \times head_{project} +I^{d\times k}\\
    map_{tail}=link_{project}\times tail_{project}+I^{d\times k}\\
    link=\argmin |head\times map_{head}+link-tail\times map_{tail}|_{L_1\mathrm{or} L_2}
\end{array}\]

TranSparse:认为map和link的异质性、link之间的不平衡性是困难所在。因此定义稀疏度sparsity$\theta$,并考虑到head和tail的稀疏度不同。
\[\begin{array}{l}
    \theta_{\min}\in[0,1],\mathrm{hyperparameter \,of \,sparsity}\\
    \theta=1-(1-\theta_{\min})N_{link}/N_{max link}\\
    link=\argmin |head\times map_{head}(\theta_{head}) + link - tail\times map_{tail}(\theta_{tail})|_{L_1\mathrm{or} L_2}
\end{array}\]

TransA:认为只用$L_1$或$L_2$距离不够灵活,且entities,map和link都每一维都被等同考虑了。因此提出用马氏距离Mahalanobis distance,它是表示点与一个分布之间的距离,并让每一维学习不同的权重。
\[\begin{array}{l}
    map: \mathrm{weight \,matrix\, related \,to\, link}\\
    link=(head+link-tail)^T \times map (head+link-tail)
\end{array}\]

TransG:不用子关系$link_{child}$来细分link,而将在每个情况下的link的不同都用一个高斯分布Normal distribution来表示。如此就可以按照分布区分出同一个link的不同作用(与细分成子link不同),从而减少错误预测。
\[\begin{array}{l}
    link=tail-head\\
    link \sim \sum\limits_{m=1}^M \pi_{link, m} N(\mu_{link, m}, I)
\end{array}\]

KG2E:认为不论是entity还是link都是不确定的。因此用高斯分布的均值表示entity或link的center,用高斯分布的协方差表示该entity或link的不确定度。并用KL距离(不对称相似度)和期望概率(对称相似度)两种方法来判定entity和link的概率相似程度。
\[\begin{array}{l}
    link=head-tail\\
    probability_{entity} \sim N (\mu_{head}-\mu_{tail}, \sum_{head}+\sum_{tail})\\
    probability_{link} \sim N(\mu_{link},\sum_{link})\\
    \forall l \in E\bigcup R, c_{\min} I\leq \sum_{l}\leq c_{\max} I,c_{\min}>0.\mathrm{to\, prevent\, overfitting}\\
    \begin{array}{lll}
        link&=&\argmin \int_{x\in \R^{k_e}} N(x;\mu_{link},\sum_{link}) \log\frac{N(x;\mu_{entity},\sum_{entity})}{N(x;\mu_{link},\sum_{link})} dx\\
        &=&\frac{1}{2} ( tr(\sum\limits_{link}^{-1}\sum\limits_{entity})\\
        && +(\mu_{entity}-\mu_{link})^T\sum\limits_{link}^{-1}(\mu_{entity}-\mu_{link})\\
        && -\log\frac{\det(\sum_{entity})}{\det(\sum_{link})}+k_e)\\
        link&=&\argmin \int_{x\in\R^{k_e}} N(x;\mu_{link},\sum_{link})N(x;\mu_{entity},\sum_{entity}) dx\\
        &=& \frac{1}{2} ((\mu_{entity}-\mu_{link})^T(\sum_{entity}+\sum_{link})^{-1}(\mu_{entity}-\mu_{link})\\
        && +\log(\det(\sum_{entity}+\sum_{link}))+k_e\log 2\pi)
    \end{array}
\end{array}\]
 
\subsection{my idea and contribution}
\subsubsection{my idea}
enrich words representation: 
sequential and spatial information,
word root and it's character length\\
本设计对relation的概念进行了调整,具体为:\\
本设计中的图谱仅由entity和运算法则组成,而不另外规定relation,以提高协同性。
本设计将relation看作由entity之间的运算而形成的,且分为两种,
一种是entities之间的渐变关系,对应map运算;
一种是entities之间的社会联系,对应link运算(就是普通kg中的relation),
用space的共键关系来表示。\\
由此,增强“实体与关系之间的联系”问题,
就可以更清楚地分为探索“实体与实体之间的map计算”和“map与link之间的计算”这两个更单纯的子问题。

在具体计算方面,本设计考虑了以下几点:\\
entity:人需用名字来记住和代表一段记忆,包括受到环境限制的认知和表现\\
实体与实体之间的map的计算关键是:认知、表现和环境限制\\
map与link之间的计算关键是:map和link同处在一个环境限制中\\
那么问题就来了:
要证明map和link的环境限制相同的话,得将二者表示出来,然后发现其中的关系。
但如果想直接认为二者的环境限制相同(而不加以证明)的话,
就应该将二者直接嵌合在一个模型中表示出来,进行embedding的学习。
如此学习到的embedding不仅可以应用于通俗文章,
也可以灵活应对各种话题、场景、专业性等环境限制的文章。
本文想做的事是后者。\\

距离就算一样,方向也会不一样。
所以map和link不能只看数量关系,也必须看spatial关系。\\
spatial关系一半用map中的hyper space来表示,另一半则用link中的probability表示。
不同的词语所面临的environmental restriction不同会导致用词的概率分布不同。\\
将topic model按照environmental restriction model来应用。
我认为二者在方法上是相通的,只是在解释性方面,我希望自己能把它的概念描述得更准确。
不同的是topic强调的是可能会出现的词语的概率,
而我想考虑的environmental restriction强调的是可能出现的词语的概率,和可能不出现的词语的概率。\\

\subsubsection{my contribution}
1. the hypothesis\\
阐述IF图,并说明应用于各行各业时可以简化或更细化的操作。

2. the notation\\
\[\begin{array}{l}
knowledge=(entities,+,\times)\\
university=(dstudent+dteacher)=(denglish+djapanese)\times map\\
\end{array}\]

\[\begin{array}{l}
p(y|x)=\frac{p(x|y)p(y)}{p(x)}\sim p(x|y)p(y)
\end{array}\]

3. the link concept\\
从过去到现在、从现有到将来的过渡关系。

\[\begin{array}{l}
(head_1, link_1, tail_1)\\
(head_2, link_2, tail_2)\\

head_1\times map_1\times map\times map_2=tail_1\\
map=\frac{tail_1}{head_1\times map_1\times map_2}\\
link_1=\tanh(map)\\

head_2\times map_2\times map\times map_1=tail_2\\
map=\frac{tail_2}{head_2\times map_2\times map_1}\\
link_2=\tanh(map)\\

link=(1-\alpha)\times link_1+\alpha\times link_2
\end{array}\]

$head_i,tail_i,map_i,link_i$ have consistency\\
$map$ in a single dataset has consistency\\

$\exists link:$\\
$head_1\times map_1\times tail_1=1$\\
$head_1\times map_1\times map\times tail_1=1$\\
$head_1\times map_1\times map\times map_2\times tail_1=1$\\
$\nexists link:$\\
$[1,512] [512, 512] [512, 512] [512, 512] [512,1]$
\end{document}